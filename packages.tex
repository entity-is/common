%
% File   : packages.tex
% Author : ɛntiˈtɛːt.kaɪ̯
% Date   : 2016-03-20
%
%---+----1----+----2----+----3----+----4----+----5----+----6----+----7----+----8



% Unterstützung für die deutsche Sprache
	\usepackage{polyglossia}

% Gleitgrenzen für Gleitumgebungen
	\usepackage{placeins}

% Ermöglicht u.a. das Hinzufügen von Seiten aus anderen PDF-Dokumenten
	\usepackage{pdfpages}

% Schöne gesetzte Titel in Gleitumgebungen
	\usepackage{caption}

% Tabellen mit vorgegebener Gesamtbreite und einer flexiblen Breite für
% Fließtextspalten.
	\usepackage{tabularx}

% Schön gesetzte Quellcode-Listings für diverse Programmiersprachen
	%\usepackage{listings}  % Wird von listingsutf8 geladen
	\usepackage{listingsutf8}

% TODOs
	\usepackage[obeyFinal,ngerman,colorinlistoftodos]{todonotes}

% Git-Metadaten in das Dokument einbinden
% Repo: https://github.com/Hightor/gitinfo2.git
	\usepackage{gitinfo2}

% Flexibler Leerraum nach Makros
	\usepackage{xspace}

% Logos aus dem TeX-Univerum
	\usepackage{hologo}

% Literaturverzeichnis
	\usepackage[backend=biber,sortlocale=auto,natbib=true,style=alphabetic]{biblatex}

% Stichwortverzeichnis
	\usepackage{imakeidx}

% Klickbare Querverweise
	\usepackage{hyperref}  % Sollte immer als letztes Paket geladen werden



%---+----1----+----2----+----3----+----4----+----5----+----6----+----7----+----8
% vim:wrap:noet:ts=2 sw=2 sts=2
