%
% File   : definitions.tex
% Author : ɛntiˈtɛːt.kaɪ̯
% Date   : 2016-03-20
%
%---+----1----+----2----+----3----+----4----+----5----+----6----+----7----+----8



% Paket „polyglossia“
	\setdefaultlanguage[
		variant=german,       % „german“ oder „austrian“
		spelling=new,         % Reformierte Trennregeln von 1996
		latesthyphen=true,    % Die neusten Trennregeln verwenden
		babelshorthands=true  % Aus „babel“ bekannte Befehle wie „"ck“,
	]                       % „"ff“, etc. behalten ihre Bedeutung
	{german}

% Ethan Schoonovers Farbschema „solarized“
% Home: http://ethanschoonover.com/solarized
% Repo: https://github.com/altercation/solarized.git
	% Background Tones
		\definecolor{Solarized-Base03}{RGB}{0,43,54}    % dark
		\definecolor{Solarized-Base02}{RGB}{7,54,66}    % dark highlights
		\definecolor{Solarized-Base2}{RGB}{238,232,213} % light highlights
		\definecolor{Solarized-Base3}{RGB}{253,246,227} % light
	% Content Tones
		% dark: comments, secondary content / light: optional emphasized content
			\definecolor{Solarized-Base01}{RGB}{88,110,117}
		% dark: - / light: body text, defaul code, primary content
			\definecolor{Solarized-Base00}{RGB}{101,123,131}
		% dark: body text, defaul code, primary content / light: -
			\definecolor{Solarized-Base0}{RGB}{131,148,150}
		% dark: optional emphasized content / light: comments, secondary content
			\definecolor{Solarized-Base1}{RGB}{147,161,161}
	% Accent Colors
		\definecolor{Solarized-Yellow}{RGB}{181,137,0}
		\definecolor{Solarized-Orange}{RGB}{203,75,22}
		\definecolor{Solarized-Red}{RGB}{220,50,47}
		\definecolor{Solarized-Magenta}{RGB}{211,54,130}
		\definecolor{Solarized-Violet}{RGB}{108,113,196}
		\definecolor{Solarized-Blue}{RGB}{38,139,210}
		\definecolor{Solarized-Cyan}{RGB}{42,161,152}
		\definecolor{Solarized-Green}{RGB}{133,153,0}

% Apple (Finder) Tag-Farben (sRGB)
	\definecolor{Finder-Tag-Red}{RGB}{255,98,92}
	\definecolor{Finder-Tag-Orange}{RGB}{255,170,71}
	\definecolor{Finder-Tag-Yellow}{RGB}{255,214,75}
	\definecolor{Finder-Tag-Green}{RGB}{131,225,99}
	\definecolor{Finder-Tag-Blue}{RGB}{78,189,250}
	\definecolor{Finder-Tag-Violet}{RGB}{214,143,231}
	\definecolor{Finder-Tag-Gray}{RGB}{164,164,167}

% Paket „tabularx“
	% Tabellenspalten mit linksbündigen Paragraphen
	% in der Schriftgröße „footnotesize“
		\newcolumntype{F}{>{\footnotesize\raggedright\arraybackslash}X}

% Paket „listingsutf8“
	\lstset{
		language=bash, %[LaTeX]tex,
		basicstyle=\tiny\ttfamily,
		showspaces=false,
		showtabs=false,
		tabsize=2,
		%keywordstyle=\bf,
		identifierstyle=\color{Solarized-Blue},
		commentstyle=\color{Solarized-Base1},
		%stringstyle=\ttfamily,
		showstringspaces=true,
		numbers=left,
		numbersep=1.5em,
		numberstyle=\tiny,
		firstnumber=1,
		numberfirstline=true,
		stepnumber=1,
		xleftmargin=3em,
		frame=trbl,
		frameround=ffff,
		framesep=1em,
		rulecolor=\color{Solarized-Base0},
		backgroundcolor=\color{Solarized-Base3},
		breaklines=true
	}

% Makros für deutschsprachige Abkürzungen
	\newcommand\zB{z.\,B.\@\xspace}  % z.B. (zum Beispiel)

% Makros für Paketnamen
	\newcommand\common{\texttt{common}\@\xspace}

% Makros für sonstige Eigennamen
	\newcommand\OSX{OS\,X\@\xspace}  % OS X (10.8+)
	\newcommand\macOSX{(Mac)\,\OSX}  % (Mac) OS X (transition from 10.7 to 10.8)
	\newcommand\MacOSX{Mac\,\OSX}    % Mac OS X (10.0-10.7)

% Zeilenabstand in Tabellen vergrößern
	\renewcommand{\arraystretch}{1.15}



%---+----1----+----2----+----3----+----4----+----5----+----6----+----7----+----8
% vim:wrap:noet:ts=2 sw=2 sts=2
